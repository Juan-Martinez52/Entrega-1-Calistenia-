\documentclass{article}
\usepackage[utf8]{inputenc}
\usepackage[spanish]{babel}
\usepackage{listings}
\usepackage{graphicx}
\graphicspath{ {images/} }
\usepackage{cite}

\renewcommand{\familydefault}{\sfdefault}

\begin{document}

\begin{titlepage}
    \begin{center}
        \vspace*{1cm}
            
        \Huge
        \textbf{Parcial 1 - Calistenia}
            
        \vspace{0.5cm}
        \LARGE
        Tarea 1
            
        \vspace{5cm}
            
        \textbf{Juan David Martinez Bonilla}
            
        \vfill
            
        \vspace{0.8cm}
            
        \Large
        Despartamento de Ingeniería Electrónica y Telecomunicaciones\\
        Universidad de Antioquia\\
        Medellín\\
        Marzo de 2021
            
    \end{center}
\end{titlepage}


\newpage
\section{Procedimiento}
A continuación se detallarán los pasos para realizar el ejercicio
de describir como llevar unos objetos de una posición A a una posición B.\\


1.	Tome la hoja con una mano\\


2.	Levante la hoja\\


3.	Con la otra mano tome las tarjetas\\


4.	Deje la hoja encima de la mesa\\


5.	Acomode/pegue las tarjetas la una con la otra\\


6.	Con la mano de preferencia tome las dos tarjetas juntas de forma vertical\\


7.	Con los dedos pulgar y medio de esa misma mano tome las dos tarjetas de las puntas del extremo superior\\


8.	Con el dedo índice mantenga presión en la parte superior (entre los dedos pulgar y medio)\\


9.	Apoye uno de los extremos inferiores de las tarjetas a la superficie encima de la hoja y ejerza presión desde arriba con el dedo índice\\


10.	Con los dedos pulgar y medio abra las tarjetas desde la parte media sin dejar de ejercer presión, con el dedo índice desde la parte superior y ladee la tarjeta que esta más cerca de la palma de su mano hacia esa misma dirección\\ 


11.	Después de llegar a la posición de equilibrio suelte con cuidado las tarjetas\\





\end{document}